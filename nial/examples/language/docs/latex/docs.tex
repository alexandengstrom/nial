\documentclass{article}
\usepackage{amsmath}
\title{Documentation}\begin{document}\maketitle\newpage\tableofcontents\newpage\section{Types}\subsection{Token}
 This type represents a token \subsubsection*{Attributes}
The type {\it Token} has the following attributes:

\begin{itemize}
\item {\bf type}:  Stores the type of the token 
\item {\bf value}:  Stores the value of the token 
\end{itemize}
\subsubsection*{Methods}
The type {\it Token} has the following methods:

\begin{itemize}\item {\bf constructor(t, v)}:  Creates the token 
\item {\bf get\_type()}: not documented
\item {\bf get\_value()}: not documented
\end{itemize}\subsection{Lexer}
 The lexer is responsible for tokenizing a string \subsubsection*{Attributes}
The type {\it Lexer} has the following attributes:

\begin{itemize}
\item {\bf available\_tokens}:  Stores a list of available match patterns 
\end{itemize}
\subsubsection*{Methods}
The type {\it Lexer} has the following methods:

\begin{itemize}\item {\bf constructor()}:  Creates the lexer 
\item {\bf define\_tokens()}:  Define all the available match patterns 
\item {\bf tokenize(string)}:  Takes a string as input, returns an token stream 
\end{itemize}\subsection{BinOpNode}
 Represents a binary operation between two other nodes \subsubsection*{Attributes}
The type {\it BinOpNode} has the following attributes:

\begin{itemize}
\item {\bf left}:  The left operand 
\item {\bf right}:  The right operand 
\item {\bf operator}:  The operator 
\end{itemize}
\subsubsection*{Methods}
The type {\it BinOpNode} has the following methods:

\begin{itemize}\item {\bf constructor(l, r, o)}:  Create the node 
\item {\bf evaluate()}:  Evaluates the left node, right node and the result of the operation 
\end{itemize}\subsection{NumberNode}
 Represents a number \subsubsection*{Attributes}
The type {\it NumberNode} has the following attributes:

\begin{itemize}
\item {\bf value}:  The value of the number 
\end{itemize}
\subsubsection*{Methods}
The type {\it NumberNode} has the following methods:

\begin{itemize}\item {\bf constructor(n)}:  Create the node 
\item {\bf evaluate()}:  Returns the value of the number 
\end{itemize}\subsection{Parser}
 Represents the parser which will create the abstract syntax tree \subsubsection*{Attributes}
The type {\it Parser} has the following attributes:

\begin{itemize}
\item {\bf tokens}:  The list of tokens we receive from the lexer 
\end{itemize}
\subsubsection*{Methods}
The type {\it Parser} has the following methods:

\begin{itemize}\item {\bf constructor(t)}:  Create the parser 
\item {\bf move\_cursor()}:  Gets the next token 
\item {\bf parse()}:  Start parsing the token stream 
\item {\bf parse\_expression()}:  Parses an expression 
\item {\bf parse\_term()}:  Parses a term 
\item {\bf parse\_factor()}:  Parses a factor 
\item {\bf parse\_binary\_operation(operators, func)}:  Parses a binary operation based on the operators and function it receives 
\end{itemize}\section{Utilities}\subsection{TokenType}
 Enum for representing token types \subsubsection*{Attributes}
The type {\it TokenType} has the following attributes:

\begin{itemize}
\item {\bf SPACE}:  A space char 
\item {\bf NUMBER}:  A number 
\item {\bf ADDITION}:  The symbol + 
\item {\bf SUBTRACTION}:  The symbol - 
\item {\bf DIVISION}:  The symbol / 
\item {\bf MULTIPLICATION}:  The symbol * 
\end{itemize}
\end{document}